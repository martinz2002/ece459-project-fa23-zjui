\documentclass[../ECE459FinalProjectReport.tex]{subfiles}

\begin{document}
\chapter{Introduction}
\section{The Background}
The history of modulation techniques dates back to the early days of radio communication when Amplitude Modulation emerged as the pioneering method. Over time, Frequency Modulation gained prominence due to its resilience against noise and superior audio quality, particularly in broadcasting and mobile communication. Both AM and FM offer distinct advantages and trade-offs, prompting the need for a comprehensive performance analysis to better understand their strengths and limitations in contemporary communication systems.

Amplitude Modulation (AM) ??? %搞个简介

Frequency Modulation 

AWGN %why use awgn to simulate noise?

\section{Problem Statement}
% Here starts the problem statement
This project involves building an AM and FM communication system and applying AWGN to this system. The team simulates the modulation and demodulation process of both systems and compares the spectrums and waveforms of the message signals and demodulated signals, where both a multi-tone message and a piece of TTS-generated voice recording are applied as the message signals. The theoretical and experimental pre- and post-detection SNRs are calculated and measured respectively for each of the systems, which allows the team to evaluate the performance of the simulated communication systems.

% why multi-tone and TTS?

% Here starts the literature review
\section{Literature Review}
% ways of building filters and getting envelopes
% also info about choosing the bandwidth
% 这里主要写理论,比如butter滤波器和Hilbert变换求包络的公式推导

\end{document}