\documentclass[../ECE459FinalProjectReport.tex]{subfiles}

\begin{document}
\chapter{Introduction}
\section{The Background}
The history of modulation techniques dates back to the early days of radio communication when Amplitude Modulation emerged as the pioneering method. Over time, Frequency Modulation gained prominence due to its resilience against noise and superior audio quality, particularly in broadcasting and mobile communication \cite[p. 152]{haykinIntroductionAnalogDigital2007}. 

Amplitude Modulation (AM) is a communication technique that transmits messages by modulating the amplitude of a radio frequency (RF) wave. This modulation is achieved through the combination of the message signal with a high-frequency carrier wave. The resulting modulated waves can be demodulated using either coherent detectors or envelope detectors.

Frequency Modulation (FM), classified as a form of Angle Modulation, involves integrating the message into the phase of an RF signal. Demodulation of the FM signal can be accomplished through the utilization of differentiators or slope circuits.

Both AM and FM present distinct advantages and trade-offs, necessitating a comprehensive performance analysis. Such an analysis is crucial for a thorough understanding of their strengths and limitations within contemporary communication systems.



\section{Objectives and Purposes}

This project is designed to conduct a comprehensive analysis of the performance of Frequency Modulation (FM) and Amplitude Modulation (AM) communication systems in the presence of noise. The methodology involves constructing an envelope-modulated AM and narrow-band FM communication system, with the subsequent application of Additive White Gaussian Noise (AWGN) to the system. Specifically, the demodulation of AM signals is to be carried out using envelope detectors.

The team is tasked with simulating the modulation and demodulation processes for both systems and subsequently comparing the spectra and waveforms of the message signals and demodulated signals. The chosen message signals encompass both a multi-tone message and a Text-to-Speech (TTS)-generated voice recording. Theoretical and experimental pre- and post-detection signal-to-noise ratios (SNRs) are to be obtained to facilitate a comprehensive performance analysis.

Furthermore, the team is required to meticulously observe disparities between input and output signals, as well as their spectra, in order to conclude the distinctive characteristics of AM and FM modulation. A comparative analysis of anti-noise performance is also imperative, involving the measurement of the signal-to-noise ratio (SNR). The evaluation of simulation performance itself is to be conducted by comparing theoretical and experimental pre- and post-detection SNRs.

This project aims to equip the team with an in-depth understanding of both the modulation and demodulation processes, enabling a nuanced comprehension of the intricacies involved in implementing communication systems using Python. Specifically, the team is expected to demonstrate proficiency in performing Fourier Transform and Hilbert Transform, as well as implementing filters and envelope detectors.


\section{Literature Review}

The textbook by \textcite[Sec. 3.1]{haykinIntroductionAnalogDigital2007} presents an intuitive method of envelope modulation. In this approach, the modulated signal is derived by combining a carrier wave with an amplified and DC-shifted message signal. For the demodulation process, the team references Haykin's work, particularly \cite[Fig. 9.8]{haykinIntroductionAnalogDigital2007}. \textcite{ulrichEnvelopeCalculationHilbert2006} introduces an alternative method of envelope detection utilizing Hilbert Transformation. This technique, when employed in conjunction with the \verb|scipy.signal| package, streamlines the implementation of envelope detector design.

The Direct Method of Frequency Modulation (FM) signal generation, as illustrated in \cite[Fig. 4.7]{haykinIntroductionAnalogDigital2007}, involves components such as an integrator, a phase modulator, and a local oscillator. The demodulation process for FM is also elucidated in the same textbook, specifically in \cite[Fig. 9.13]{haykinIntroductionAnalogDigital2007}.

The realization of ideal filters poses challenges due to the discontinuous frequency response. However, the Butterworth filter offers a practical solution for simulating real-world filters, as discussed in the works of Storr \cite{storrButterworthFilterDesign2013} and Khetarpal et al. \cite{khetarpalZaiPythonZhongShiXianDiTongLuBoQi2022}. Implementation of the Butterworth filter can be achieved using the \verb|scipy.signal| package.


\end{document}